\section*{Úvod a~cíle}
\addcontentsline{toc}{section}{\protect\numberline{}Úvod a~cíle}
Mým vybraným tématem pro seminární práci z~předmětu 3PE112 -- Podniková ekonomika pro informatiky a~statistiky je zakladatelský rozpočet pro zamýšlený podnik. Při jedné návštěvě v~mé rodině zazněl originální nápad o~založení firmy, která bude provozovat rozvoz vietnamských jídel, jelikož v~rodině máme prokazatelně šikovné kuchaře. Dalším důvodem bylo, že v~současné době, kdy se stává vietnamská kuchyně velmi oblíbenou stále více a~více lidí v~České republice, bylo by fajn využít této příležitosti k~novému podnikání. Cílem této práce je sestavit podnikový finanční rozpočet, ze kterého lze poznat, jestli je pro případ založení podniku reálné a~smysluplné provozovat zmíněnou službu.





\section{Podnikatelský záměr}
Budeme tedy provozovat cateringovou službu, která bude mít název \uv{Papajas Delivery}. Hlavní činnosti služby jsou vaření a~rozvoz čerstvých exotických obědů po hl. m. Praze. Chceme nabízet rozvoz obědů pracovníkům, kteří nemají čas se naobědvat v~poledne. Budeme se specializovat především v~kvalitní výrobě a~rovněž rychlém a~spolehlivém dodání cenově dostupných výrobků, které přispívají ke zdravé výživě a~jsou základem pro zdravou stravu, dostatek energie pro práci stále většího počtu zájemců v~Praze.

\subsection{Zaměření}
Z důvodu nedostatku pracovních sil zpočátku bude služba zaměřená pouze na velká kancelářská střediska, jako jsou Skanska, Chodov Office Park, Kavčí hory, Gemini atp. Většina velkých kancelářských center se nachází v~Praze 4, proto si zvolíme kancelář a~místo vaření na okraji jižně od Prahy, konkrétně v~Jesenicích, odtud je velmi dobrá doprava k~vybraným destinacím, jelikož budeme rozvážet vlastním nakoupeným automobilem. 

Pro začínající podnik je důležité optimalizovat náklady a~tím minimalizovat případnou ztrátu. Nájem provozního prostoru na okraji Prahy je proto dobrým tahem a taky protože bude potřeba vestavět novou kuchyň v~na místě, z~čehož vyplývá nákup kuchyňských potřeb.

\subsection{Provoz}
Procházíme informační dobou. Víme, že originální nápad a~technologie jsou nezbytným základem úspěchu, proto budeme využívat výhody internetu. Objednávky se budou přijímat prostřednictvím webové aplikace, která bude rovněž poskytovat důležité informace pro náš podnik. Objednávky přes telefon jsou též možné. Zdrojem příjmů tedy budou objednávky obědů, jejichž ceny budou zahrnovat i~dopravné.

Budeme potřebovat minimálně tři pracovníky. Jednoho kuchaře, jednoho pomocníka a~jednoho až dva řidiče na rozvoz. V prvním roce po zahájení budeme předpokládat tři zaměstnance, s~tím, že denně budeme očekávat množství objednávek v~rozmezí 50 až 100 obědů. Ceny obědů budou stanoveny v~další kapitole.





\section{Počátek}
Před zahájením si budeme muset zajistit prostor a~kuchyňská vybavení pro vaření. Cena pronájmu prostoru o~celkové velikosti 80~m$^2$ by činila 14~000~Kč/měsíc. Prostor bude dostatečně velká pro kuchyň i~menší kancelář. Kancelář bude sloužit k~sledování objednávek a~vytištění informací pro pracovníky. Předpokládaná cena za využití plynu, elektřiny a~vody je díky nenáročnosti provozu přijatelná, a~to 8~000~Kč/měsíc. Cena za ostatní služby a~náležitosti (odpad, zabezpečení, účetnictví, \ldots) by činila 3~000~Kč/měsíc.

\subsection{Náklady}
Náš prostor bude pochopitelně potřebovat úpravu (odvzdušení, přívod plynu, vody) a~ná\-sledně naplnit nábytkem a~kuchyňskými spotřebiči. Do kanceláře potřebujeme internet, počítač a~tiskárnu. Dále je potřeba pro rozvoz pořídit dodávkové auto. Následující tabulka popisuje seznam předpokládaných potřebných investic do naší kuchyně a~rozvozu (v Kč).

\textbf{TABULKA 1}

\subsection{Pracovní doba a~platy zaměstnancům}
Budeme se zaměřovat pouze na dodání obědů. Objednávky musí být zadány alespoň do 17h den předem. To znamená, že vařit se bude pouze ráno a~rozvážet se bude v~poledne. Odpoledne se bude uklízet a~připravovat na další den. Jelikož se o~víkendu většinou nepracuje, objednávky se budou přijímat pouze na pondělí až pátek. Pracovní doba tedy bude cca od 7h do 15h, tedy 8 hodin denně, 5 dní v~týdnu. Nebudeme uvažovat svátky.

Kuchař bude mít vzhledem k~jeho zkušenostem největší plat ze všech, a~to 30~000~Kč/měsíc. Pomocník 15~000~Kč/měsíc. Řidič 15~000~Kč/měsíc. Mzdy budou takto stanoveny, neboť budou všichni zaměstnanci se navzájem pomáhat. Řidič nebude pouze rozvážet, ale také se bude podílet na přípravu jídel (např. balení).





\section{Zahájení činnosti}
Náklady, které budou jednorázově vynaloženy před zahájením, vyplývá z~výše uvedené tabulky v kapitole 2. Celkem tedy 785~000~Kč. Dále musíme počítat i~s~ měsíčními náklady, tzn. provozní a~mzdové náklady.

\textbf{TABULKA 2}

\subsection{Počáteční rozvaha}
Jsme schopni do podniku vložit 1~200~000~Kč vlastního kapitálu. Zbytek financí, na který je třeba pořídit bankovní úvěr, můžeme dopočítat z~rozvahy následovně:

\textbf{TABULKA 3}

Peníze na bankovním účtu slouží k~placení mezd, pojištění, faktur za energie a~služby, zboží, úvěrových splátek, rekonstrukce. Účet také bude sloužit jako účet pro bezhotovostní platby zákazníků.

\subsection{Bankovní úvěr}
Je zřejmé z~rozvahy, že budeme muset najít půjčku za 700~000~Kč. Šlo by o~relativně malý úvěr, který bychom mohli rychle zajistit v~kterékoli bance. Jelikož máme dobrou zkušenost s~Českou spořitelnou~a.~s., budeme u~nich hledat. V současné době ČSAS nabízí výhodné rychlé půjčky. Podle kalkulačky na stránkách České spořitelny by se úroková sazba za půjčku 700~000~Kč na dobu 5 let mohla činit 5,8~\% a~měsíční splátka by byla konstantních 13~881~Kč. Roční vývoj splácení (umořovací schéma), ze kterého jsou patrné roční úrokové náklady, bude vypadat následovně:

\textbf{TABULKA 4}

V prvním roce tedy bude činit splátka 180~600~Kč. Ta se bude skládat ze 40~600~Kč úroku a~140~000 Kč úmoru.





\section{Vývoj podnikání}
Jediným zdrojem příjmů jsou objednávky obědů. Naše služba bude nabízet na webových stránkách pestré menu. V tomto okamžiku již máme vymyšlených přes 20 jídel. Ceny se budou pohybovat v~rozmezí 60 až 110~Kč včetně dopravné. Drtivou většinu pokrmů ale budou tvořit 110~Kč položky, tj. hlavní jídla (rýže, pho, nudle). Ukázkové menu:

\textbf{TABULKA 5}

\subsection{Tržby}
Přestože je náš podnikatelský záměr originální, situace trhu je bohužel pro nás zatím nepříznivá, protože vaříme asijské obědy v~evropské zemi. Znamená to, že i~když se budeme snažit vynikat kvalitou provedení a~ručit zdravou výživu, naše produkty nemusejí být vítány českou komunitou, protože popularita exotických pokrmů v~současné době není příliš velká, aneb jen začíná stoupat. Z tohoto důvodu budeme v~prvním roce provozu předpokládat pomalou růst ve výnosu. V letních měsících mohou tržby výrazně klesat kvůli hromadným dovoleným. Přibližný vývoj tržeb v~prvních měsících fungování bude naznačen na následujícím grafu.

\textbf{Obrázek 1}

Výše tržeb vycházejí z~počtu objednávek. Budeme uvažovat pouze 110~Kč objednávky. Denně by se dalo očekávat celkem 50 až 100 objednávek, měsíčně tedy cca přes 1~300 obědů a~průměrná denní tržba by byla přes 8~000~Kč. Záleží také ale na tom, v~jakém měsíci budeme začínat. Na předchozím grafu je znázorněn případ zahájení činnosti v~lednu. Bude chvíli trvat, než se dostaneme do fáze stability. V prvních měsících bude pro nás ztráta bohužel nevyhnutelná (tzv. hladová trasa fáze růstu podniku, kdy jsou výdaje výrazně větší než příjmy).

\subsection{Daňové odpisy}
Odepisovat budeme všechen dlouhodobý majetek, jelikož vše bylo pořízeno nově. Skládá se z~dodávkového automobilu a~vybavení v~pronajatém prostoru a~webové aplikace jako nehmotný dlouhodobý majetek. Nábytek, nádobí a~pomůcky činí celkem 60~000~Kč, stroje 85~000~Kč, software 60~000~Kč. Odpisová sazba vychází z~následující oficiální tabulky:

\textbf{Tabulka 6}

Pro dlouhodobý nehmotný majetek, tedy náš objednávkový systém (webová aplikace), kterou bychom koupili za 60~000~Kč, bude platit rovnoměrný 3letý odpis (jelikož 60~000~Kč je dolní hranice vstupní ceny nehmotného majetku). Ročně tedy činí odpis softwaru $60~000 \div 3 = 20~000~\text{Kč}$. Nebudeme počítat s~žádným technickým zhodnocením (upgrade).

V prvním roce budeme aplikovat rovnoměrné odpisy následovně:

\textbf{Tabulka 7}





\section{Výsledky}

Po prvním roce se očekává účetní výsledek, zda se bude jednat o zisk či ztrátu. Buďme optimističtí.

\subsection{Výsledovka ke konci prvního roku po zahájení}
Sazba daně z~příjmů fyzických osob v~České republice činí 15~\%. Daň musíme pak na konci ještě odečíst z~výsledku hospodaření, abychom měli čistý zisk (EAT).

\textbf{Tabulka 8}

\subsection{Rozvaha ke konci prvního roku po zahájení}
Nájemné se bude platit na fakturu a~vždy do 15.~dne následujícího měsíce bezhotovostním převodem. To samé bude platit u~faktur za energie, služby, doména, hosting, internet, mobilní tarif a~pojištění rizik. Mzdy budou vypláceny též měsíc zpětně, a~to bezhotovostně na účty zaměstnanců, současně se bude platit i~zdravotní a~sociální pojištění k~danému měsíci. Zásobu a~materiál budeme mít zajištěné každý měsíc od dodavatele asijských potravin a~gastronomicky specializovaného dodavatele a~to vše na fakturu se splatností 14~dnů. Hotovost slouží mimo jiné hlavně k~menších nákupů zelenin, syrového masa a~přísad ve velkoobchodech. V průběhu roku by se určitě muselo dojít k~vkládání hotovosti z~tržeb na bankovní účet, ze kterého budou odcházet průběžné výdaje.

\textbf{Tabulka 9}

Pasivum dodavatelé zahrnuje veškeré nezaplacené faktury za služby či zboží včetně nájemné, tzn. závazky k~dodavatelům.

\subsection{Rozvaha ke konci prvního roku po zahájení}
Přímou i~nepřímou metodou:

\textbf{Tabulka 10}




\section{Finanční páka a~ukazatelé rentability}
EBIT by nám vyšel $149~744 + 26~426 + 40~600 = 216~770~\text{Kč}$. Nyní si spočítáme, jak přínosný by pro nás byl úvěr za 700~000~Kč od České spořitelny.

Je zde patrné, že rentabilita vlastního kapitálu by byla u~varianty s~úvěrem vyšší než u~varianty bez úvěru. Finanční páka by působila na náš podnik pozitivně, i~když ne o~moc (cca 3~\%) výhodnější.

Rentabilita vlastního kapitálu (ROE) by byla cca 12,5~\% a~rentabilita aktiv by byla $ \text{ROA} = \text{zisk} \div \text{aktiva} = 149~744 \div 2~101~570 = 7,12~\%$. Hodnoty ukazatelů rentability jsou pochopitelně nízké, jelikož jsme začínající podnik a~máme za sebou jen první \uv{zdařilý} rok podnikání.





\section{Oportunitní náklady}
\textit{Last but not least}. Zajímalo by nás, co kdybychom do tohoto podnikání nešli, je podnikání pro nás přínosnější, než když bychom pracovali jinde či investovali jinak? To si také můžeme spočítat na základě tzv. \textit{oportunitních nákladech}, pomocí kterých lze také zjistit, zda firma dosahuje ekonomický zisk.

Mezi oportunitními náklady lze zařadit mzdu vlastníka, kdyby pracoval v~jiné firmě. Dále jeho finanční úsporu, kterou by byl schopen vložit do kapitálu firmy. Tu by mohl naspořit v~bance, nejlépe přes dluhopisy. Průměrný výnos z~dluhopisů by byl cca 5~\% za rok.

\begin{itemize}
\item měsíční plat vlastníka u~jiné firmy: 25~000~Kč
\item úspora vlastníka: 1~200~000~Kč
\end{itemize}

Celkem roční oportunitní náklady: $1~200~000 \times 0,05 + 12 \times 25~000 = 360~000~\text{Kč}$. Pokud by byl vlastník přímo kuchařem, jeho roční příjem by byl $12 \times 30~000 + 149~744 = 509~744~\text{Kč}$, což je patrně vyšší plat než u~starého zaměstnání. Je třeba konečně zde Vám čtenáři prozradit, že podnikatelský záměr byl skutečně nápad mého strýce, který pracuje jako profesionální šéfkuchař v~PHO Original restauraci v~Praze.





\section*{Závěr}
\addcontentsline{toc}{section}{\protect\numberline{}Závěr}
\textit{Easier said than done}. K takovému optimisticky kladnému výsledku hospodaření v~prvním roce podnikání se chtěl dopracovat každý. Aby to ale vše mohlo skutečně vyjít, alespoň tedy přibližně, je potřeba do našeho podnikového plánu zapojit mnohem více věcí a~úsilí, např. marketing je klíčový pro rozjezd služby, klást si důraz na kvalitu. Ze začátku je zásadní propagovat naši službu s~maximálním úsilím. Skutečnost je, že pokud bychom získali alespoň 20~\% zájemců o~naši službu přes rozdávání letáků před kanceláří, bylo by to i~tak velké štěstí. Pomohlo by mimo jiné i~propagace přes internet, tj. na sociálních sítích (Facebook, Twitter) a~pomocí reklamních kampaní (Google AdWords) či doporučení přes populární gastronomické blogy.

Nesmíme zapomenout, že naše služba je poháněna hlavně za pomocí webové aplikace. Je to místo, kde se odehrávají nejdůležitější pohyby, tj. styky se zákazníky. Je proto velmi důležité se jí věnovat a~zdokonalit jak funkčnost, tak i~nefunkční požadavky a~případně posílit podporování zákazníků. Časem by bylo dobré najmout dalšího pracovníka, který se bude starat o~CRM. V případě úspěchu v~prvním roce by se firmě vyplatilo nechat zhotovit mobilní aplikaci pro zpevnění vztahu se zákazníky pomocí notifikací atp. Rozhodně není zbytečné naše menu do budoucna rozšířit o~nové atraktivnější pokrmy.