%% Zpracujte dokument na libovolné téma s respektováním dále uvedených požadavků:
% 1. Povinné části dokumentu: titulní strana (vlastní úprava), obsah, text, seznam použitých pramenů, rejstřík.
% 2. Práce s použitými prameny bude založena na samostatné bibliografické databázi.
% 3. V dokumentu bude použita plovoucí grafika/tabulka.
%% 4. V cílově generovaném PDF budou záložky (bookmarks).
%% 5. Rozvržení stran bude navrženo pro oboustranný tisk. Doporučené velikosti okrajů: vnitřní 20mm, horní 30mm, vnější 40mm a dolní 50mm. Pro nastavení okrajů lze s výhodou použít balíček geometry.
%% 6. Do horního záhlaví umisťujte vždy název aktuálně nejvyššího nadpisu tak, aby byl zarovnán k vnějšímu okraji. Číslo strany umisťujte do paty strany tak, aby bylo zarovnáno na vnější okraj. Pro úpravu obsahu záhlaví a zápatí lze doporučit balíček fancyhdr.
% 7. Dokument musí být obsahově soudržný.

% Do odevzdávárny vložte kompletní sadu souborů tvořících logicky daný dokument - včetně vygenerovaného výsledku.

\documentclass[a4paper,11pt,twoside]{article}

\usepackage[czech]{babel}
\usepackage[fixlanguage]{babelbib}
\selectbiblanguage{czech}
\usepackage[utf8]{inputenc}
\usepackage[%hidelinks,
            unicode]{hyperref}
\usepackage[%showframe,
            headheight=14.6pt,
            top=3cm,bottom=4.5cm,
            inner=4cm,outer=2cm]{geometry}
\usepackage{graphicx}
\usepackage{fancyhdr}
\usepackage{lipsum}
\usepackage{bookmark}
\usepackage{setspace}
\usepackage{amsmath}
\usepackage{pgfplots}
\pgfplotsset{compat=1.12}

\usepackage{makeidx}

% using texindy
% \makeindex

\newcommand\blankpage{
    \null
    \thispagestyle{empty}
    \newpage
}
\onehalfspacing
\begin{document}
\def\arraystretch{1.3}
%----------------------------------------------------------------------------------------
%	TITLE PAGE
%----------------------------------------------------------------------------------------
%\newgeometry{margin=3cm} % override default margins
\begin{titlepage}
\begin{center}

\textsc{\LARGE Vysoká škola ekonomická v Praze}\\[0.5cm]
\textsc{\large Fakulta informatiky a statistiky}\\[0.5cm]
\textsc{\large Katedra informačního a znalostního inženýrství}\\[0.5cm]

\vfill

\hrule
\vspace{0.5cm}
\huge {\bfseries Zakladatelský rozpočet}
\vspace{0.4cm}
\hrule
\vspace{0.4cm}
\large Seminární práce \\ 3PE112 -- Podniková ekonomika pro informatiky a statistiky\\[0.3cm]

\vfill

\large Nguyen Viet Bach \hfill 20. prosince 2015

\end{center}
\end{titlepage}
%\restoregeometry % reset to default margins


\blankpage
%\pagenumbering{gobble}
%%----------------------------------------------------------------------------------------
%	ACKNOWLEDGMENTS PAGE
%----------------------------------------------------------------------------------------

\vspace*{\fill}
\section*{Prohlášení}
%\addcontentsline{toc}{section}{\protect\numberline{}Prohlášení}%
\lipsum[1]
\lipsum[2]
\vspace{2cm}
\begin{flushright}
\underline{\hspace{5cm}}
\end{flushright}
V Praze dne \today \hfill Nguyen Viet Bach
\newpage
 \blankpage
%%----------------------------------------------------------------------------------------
%	ACKNOWLEDGMENTS PAGE
%----------------------------------------------------------------------------------------

\vspace*{\fill}
\section*{Poděkování}
%\addcontentsline{toc}{section}{\protect\numberline{}Poděkování}%
\lipsum[2]

\newpage
 \blankpage
\pagenumbering{roman}
%\setcounter{page}{7}
\setcounter{page}{3}
%%----------------------------------------------------------------------------------------
%	ABSTRACT PAGE
%----------------------------------------------------------------------------------------

\section*{Abstrakt}
%\addcontentsline{toc}{section}{\protect\numberline{}Abstrakt}%
\lipsum[1]
\lipsum[2]
\lipsum[3]

\newpage
 \blankpage

\fancyhf{}
\pagestyle{fancy}
\fancyfoot[CO,CE]{\thepage}
%----------------------------------------------------------------------------------------
%	TABLE OF CONTENTS
%----------------------------------------------------------------------------------------

\tableofcontents

\newpage
\blankpage

\fancyhf{}
\pagestyle{fancy}
\fancyfoot[RO,LE]{\thepage}
\fancyhead[RO,LE]{\leftmark}
\pagenumbering{arabic}

\section{Úvod}
%\addcontentsline{toc}{section}{\protect\numberline{}Úvod}%

\lipsum

\newpage

\section{Terminologie}

\lipsum

\cite{einstein} Hell yeah.

\newpage

\section{Polymorphism}

\lipsum

\newpage

\section{Inheritance}

\lipsum

\subsection{Cats}

\lipsum

\newpage


\section{Lambda}

\lipsum

\newpage




\fancyhf{}
\pagestyle{fancy}
\bibliography{references}
\bibliographystyle{babplain}
\addcontentsline{toc}{section}{Reference}
\newpage

\addcontentsline{toc}{section}{Seznam obrázků}
\listoffigures

\addcontentsline{toc}{section}{Seznam tabulek}
\listoftables
\newpage

% the following 3 lines set the index page's format to fancy
% otherwise this page would have plain style
\makeatletter
\let\ps@plain\ps@fancy
\makeatother
\addcontentsline{toc}{section}{Rejstřík}
\printindex

\newpage

\end{document}  
